%% This is the ctufit-thesis example file. It is used to produce theses
%% for submission to Czech Technical University, Faculty of Information Technology.
%%
%% Get the newest version from
%% https://gitlab.fit.cvut.cz/theses-templates/FITthesis-LaTeX
%%
%%
%% Copyright 2021, Eliska Sestakova and Ondrej Guth
%%
%% This work may be distributed and/or modified under the
%% conditions of the LaTeX Project Public Licenese, either version 1.3
%% of this license or (at your option) any later version.
%% The latest version of this license is in
%%  https://www.latex-project.org/lppl.txt
%% and version 1.3 or later is part of all distributions of LaTeX
%% version 2005/12/01 or later.
%%
%% This work has the LPPL maintenance status `maintained'.
%%
%% The current maintainer of this work is Ondrej Guth.
%% Contact ondrej.guth@fit.cvut.cz for bug reports.
%% Alternatively, submit bug reports into the tracker at
%% https://gitlab.fit.cvut.cz/theses-templates/FITthesis-LaTeX/issues
%%
%%

%%%%%%%%%%%%%%%%%%%%%%%%%%%%%%%%%%%%%%%%%
% CLASS OPTIONS
% language: czech/english/slovak
% thesis type: bachelor/master/dissertation
%%%%%%%%%%%%%%%%%%%%%%%%%%%%%%%%%%%%%%%%%
\documentclass[czech,master,unicode]{ctufit-thesis}

%%%%%%%%%%%%%%%%%%%%%%%%%%%%%%%%%%
% FILL IN THIS INFORMATION
%%%%%%%%%%%%%%%%%%%%%%%%%%%%%%%%%%
\ctufittitle{Heuristické přístupy při řešení NP-těžkých úloh} % replace with the title of your thesis
\ctufitauthorfull{Bc. Tomáš Petříček} % replace with your full name (first name(s) and then family name(s) / surname(s)) including academic degrees
\ctufitauthorsurnames{Petříček} % replace with your surname(s) / family name(s)
\ctufitauthorgivennames{Tomáš} % replace with your first name(s) / given name(s)
\ctufitsupervisor{Ing.\,Michal Šoch,\,Ph.D.} % replace with name of your supervisor/advisor (include academic degrees)
\ctufitdepartment{Katedra softwarového inženýrství} % replace with the department of your defence
\ctufityear{2023} % replace with the year of your defence
\ctufitdeclarationplace{Praze} % replace with the place where you sign the declaration
\ctufitdeclarationdate{\today} % replace with the date of signature of the declaration
\ctufitabstractCZE{
Cílem práce je nalézt téměř optimální konfiguraci pro algoritmickou obchodní strategii za pomocí heuristik v rozumně krátkém čase.
Čtenář je nejprve seznámen se třídami složitosti a heuristickými metodami, včetně těch použitých v praktické části: simulované ochlazování,
genetický algoritmus a tabu prohledávání.
Následuje popis algoritmického obchodování a fází, které lze automatizovat.

Znalosti z obou těchto oborů jsou dány dohromady v sekci zaměřené na zpětné testování.
Cílem zpětného testování je otestovat obchodní strategii na historických tržních datech, aby bylo možné provést úpravy a najít vhodnou konfiguraci předtím, něž je použita pro skutečné testování.
Nalezení vhodné konfigurace je NP-těžký problém, nicméně lze použít heuristiky pro jeho nalezení v rozumném čase.

Za účelem dosažení větší flexibility a rychlosti byl vytvořen vlastní framework napsaný v C++20.
Historická simulace byla inspirována platformou TradingView a proces obchodování burzou Binance.
Byly použity moderní prvky C++, jako jsou koncepty a korutiny.
Důraz byl kladen na efektivní využití funkcí jazyka a prostředků počítače.

Kromě heuristiky byl implementován také algoritmus hrubé síly, který byl paralelizován pomocí OpenMP k dosažení vyšší rychlosti.
Heuristiky byly implementovány tak, aby byly flexibilní a snadno použitelné, proto byly použity návrhové vzory jako pozorovatel a strategie.
Průběh a výsledky shromážděné během optimalizace byly uloženy do souborů ve formátu JSON a CSV, aby mohly být později zpracovány pomocí vědeckých balíčků jazyka Python.

Framework byl unit testován před provedením experimentálního vyhodnocení.
Pro experimenty byly použity minutové svíčky 10 různých kryptoměn.
Nejprve byl vybrán vyhledávací prostor s rozumnými konfiguracemi a velikostí.
Nejlepší konfigurace byly nalezeny pomocí algoritmu hrubé síly.
Následně byly provedeny white-box a black-box testy pomocí heuristických přístupů a výsledky porovnány s optimálními.
Heuristické přístupy jsou schopny najít téměř optimální konfigurace během několika minut namísto hodin nebo dokonce dnů pomocí i nejoptimalizovanějšího algoritmu hrubé síly.
}
\ctufitabstractENG{
The aim of the thesis is to find a near optimal configuration for an algorithmic trading strategy using heuristics in reasonably short time.
The reader is firstly introduced to complexity classes and heuristic methods, including those used in the practical part: simulated annealing, genetic algorithm and tabu search.
It is followed by the description of algorithmic trading and the stages that can be automated.

Knowledge from both of these fields is combined in the backtesting section.
The goal of backtesting is to test a trading strategy against historical data to make adjustments and find appropriate configuration before deciding whether or not to use it in real-time trading.
Finding these configurations is an NP-hard problem, and heuristics can be used to find the optimal ones in a reasonable time.

In order to achieve greater flexibility and speed, a custom framework written in C++20 was created.
The historical simulation was inspired by the TradingView platform and the trading process by the Binance exchange.
Modern C++ features such as concepts and coroutines were used.
The emphasis was placed on the effective use of the language features and computer resources.

In addition to heuristics, a brute force algorithm was also implemented.
It was parallelized using OpenMP to achieve higher speed.
Heuristics were implemented to be flexible and easy to use, thus design patterns such as Observer and Strategy were used.
The progress and results collected during the optimization were saved in JSON and CSV files, so that they can later be examined using scientific Python packages.

The framework was unit tested before the experimental evolution was performed.
Years of one-minute candlestick data of 10 different cryptocurrencies were used for the evaluation.
First, a search space with reasonable configurations and size was selected.
The best configurations were found using a brute force algorithm.
White-box tests and black-box tests were then performed using heuristic approaches, and the results were compared with the optimal ones.
Overall, heuristic approaches are able to find near-optimal configurations in minutes instead of hours or even days using even the most optimized brute-force algorithm.
}
\ctufitkeywordsCZE{zpětné testování, optimalizace, metaheuristiky, algoritmické obchodování}
\ctufitkeywordsENG{backtesting, optimization, metaheuristics, algorithmic trading}
%%%%%%%%%%%%%%%%%%%%%%%%%%%%%%%%%%
% END FILL IN
%%%%%%%%%%%%%%%%%%%%%%%%%%%%%%%%%%

%%%%%%%%%%%%%%%%%%%%%%%%%%%%%%%%%%
% CUSTOMIZATION of this template
% Skip this part or alter it if you know what you are doing.
%%%%%%%%%%%%%%%%%%%%%%%%%%%%%%%%%%

\RequirePackage{iftex}[2020/03/06]
\iftutex % XeLaTeX and LuaLaTeX
    \RequirePackage{ellipsis}[2020/05/22] %ellipsis workaround for XeLaTeX
\else
    \RequirePackage[utf8]{inputenc}[2018/08/11] %this file encoding
    \RequirePackage{lmodern}[2009/10/30] % vector flavor of Computer Modern font
\fi

% hyperlinks
\RequirePackage[pdfpagelayout=TwoPageRight,colorlinks=false,allcolors=decoration,pdfborder={0 0 0.1}]{hyperref}[2020-05-15]

% uncomment the following to hide all hyperlinks
% \RequirePackage[pdfpagelayout=TwoPageRight,hidelinks]{hyperref}[2020-05-15]

\RequirePackage{pdfpages}[2020/01/28]

\setcounter{secnumdepth}{4} % numbering sections; 4: subsubsection

\usepackage{multirow}
\usepackage{tabularray}
\usepackage{algorithm}
\usepackage{algpseudocode}

%%%%%%%%%%%%%%%%%%%%%%%%%%%%%%%%%%
% CUSTOMIZATION of this template END
%%%%%%%%%%%%%%%%%%%%%%%%%%%%%%%%%%


%%%%%%%%%%%%%%%%%%%%%%
% DEMO CONTENTS SETTINGS
% You may choose to modify this part.
%%%%%%%%%%%%%%%%%%%%%%
\usepackage{dirtree}
\usepackage{lipsum,tikz}
\usepackage{csquotes}
\usepackage[style=iso-numeric]{biblatex}
\addbibresource{text/bib-database.bib}
\usepackage{listings} % typesetting of sources
% \usepackage{minted} % typesetting of sources

%theorems, definitions, etc.
\theoremstyle{plain}
\newtheorem{theorem}{Věta}
\newtheorem{lemma}[theorem]{Tvrzení}
\newtheorem{corollary}[theorem]{Důsledek}
\newtheorem{proposition}[theorem]{Návrh}
\newtheorem{definition}[theorem]{Definice}
\theoremstyle{definition}
\newtheorem{example}[theorem]{Příklad}
\theoremstyle{remark}
\newtheorem{note}[theorem]{Poznámka}
\newtheorem*{note*}{Poznámka}
\newtheorem{remark}[theorem]{Pozorování}
\newtheorem*{remark*}{Pozorování}
\numberwithin{theorem}{chapter}
%theorems, definitions, etc. END
%%%%%%%%%%%%%%%%%%%%%%
% DEMO CONTENTS SETTINGS END
%%%%%%%%%%%%%%%%%%%%%%

\begin{document} 
\frontmatter\frontmatterinit % do not remove these two commands

\includepdf[pages={1-}]{petrito6-assignment.pdf} % replace that file with your thesis assignment provided by study office

\thispagestyle{empty}\cleardoublepage\maketitle % do not remove these three commands

\imprintpage % do not remove this command

\tableofcontents % do not remove this command
%%%%%%%%%%%%%%%%%%%%%%
% list of other contents: figures, tables, code listings, algorithms, etc.
% add/remove commands accordingly
%%%%%%%%%%%%%%%%%%%%%%
\listoffigures % list of figures
\begingroup
\let\clearpage\relax
\listoftables % list of tables
\lstlistoflistings % list of source code listings generated by the listings package
% \listoflistings % list of source code listings generated by the minted package
\endgroup
%%%%%%%%%%%%%%%%%%%%%%
% list of other contents END
%%%%%%%%%%%%%%%%%%%%%%

%%%%%%%%%%%%%%%%%%%
% ACKNOWLEDGMENT
% FILL IN / MODIFY
% This is a place to thank people for helping you. It is common to thank your supervisor.
%%%%%%%%%%%%%%%%%%%
\begin{acknowledgmentpage}
Zvláště bych chtěl poděkovat svému vedoucímu za naslouchání během konzultací, vedení práce a schopnosti mě přimět se soustředit na to podstatné.
Rád bych také poděkoval Petru Tešnarovi za poskytnutí obchodní strategie a uvedení do světa algoritmického obchodování.

\end{acknowledgmentpage} 
%%%%%%%%%%%%%%%%%%%
% ACKNOWLEDGMENT END
%%%%%%%%%%%%%%%%%%%


%%%%%%%%%%%%%%%%%%%
% DECLARATION
% FILL IN / MODIFY
%%%%%%%%%%%%%%%%%%%
% INSTRUCTIONS
% ENG: choose one of approved texts of the declaration. DO NOT CREATE YOUR OWN. Find the approved texts at https://courses.fit.cvut.cz/SFE/download/index.html#_documents (document Declaration for FT in English)
% CZE/SLO: Vyberte jedno z fakultou schvalenych prohlaseni. NEVKLADEJTE VLASTNI TEXT. Schvalena prohlaseni najdete zde: https://courses.fit.cvut.cz/SZZ/dokumenty/index.html#_dokumenty (prohlášení do ZP)
\begin{declarationpage}
Prohlašuji, že jsem předloženou práci vypracoval samostatně a že jsem uvedl veškeré použité
informační zdroje v souladu s Metodickým pokynem o dodržování etických principů při přípravě
vysokoškolských závěrečných prací.

Beru na vědomí, že se na moji práci vztahují práva a povinnosti vyplývající ze zákona č. 121/2000 Sb.,
autorského zákona, ve znění pozdějších předpisů, zejména skutečnost, že České vysoké učení
technické v Praze má právo na uzavření licenční smlouvy o užití této práce jako školního díla podle §
60 odst. 1 citovaného zákona.
\end{declarationpage}
%%%%%%%%%%%%%%%%%%%
% DECLARATION END
%%%%%%%%%%%%%%%%%%%

\printabstractpage % do not remove this command

%%%%%%%%%%%%%%%%%%%
% SUMMARY
% FILL IN / MODIFY
% OR REMOVE ENTIRELY (upon agreement with your supervisor)
% (appropriate to remove in most theses)
%%%%%%%%%%%%%%%%%%%
% \begin{summarypage}
% \section*{Summary section}

% \lipsum[1][1-8]

% \section*{Summary section}

% \lipsum[2][1-6]

% \section*{Summary section}

% \lipsum[3]

% \section*{Summary section}

% \lipsum[2]

% \section*{Summary section}

% \lipsum[1][1-8] Lorem lorem lorem.
% \end{summarypage}
%%%%%%%%%%%%%%%%%%%
% SUMMARY END
%%%%%%%%%%%%%%%%%%%

%%%%%%%%%%%%%%%%%%%
% ABBREVIATIONS
% FILL IN / MODIFY
% OR REMOVE ENTIRELY
% List the abbreviations in lexicography order.
%%%%%%%%%%%%%%%%%%%
\chapter{Seznam zkratek}
	
\begin{tabular}{rl}
CSV & Comma-Separated Values\\
EMA & Exponential Moving Average\\
JSON & JavaScript Object Notation\\
PROM & Pessimistic Return On Margin\\
SMA & Simple Moving Average\\
\end{tabular}
%%%%%%%%%%%%%%%%%%%
% ABBREVIATIONS END
%%%%%%%%%%%%%%%%%%%

\mainmatter\mainmatterinit % do not remove these two commands

%%%%%%%%%%%%%%%%%%%
% THE THESIS
% MODIFY ANYTHING BELOW THIS LINE
%%%%%%%%%%%%%%%%%%%

% Do not forget to include Introduction
%---------------------------------------------------------------
% \chapter{Introduction}
% uncomment the following line to create an unnumbered chapter
\chapter*{Introduction}\addcontentsline{toc}{chapter}{Introduction}\markboth{Introduction}{Introduction}
%---------------------------------------------------------------
\setcounter{page}{1}

Backtesting je nezbytnou součástí při vývoji obchodní strategie.
Ukazuje, jak by si strategie vedla, kdyby byla nasazena v minulosti.
Cílem je lépe pochopit fungování dané strategie, zjistit, kde jsou její úskalí a ideálně je opravit před nasazením do produkce.
Strategie jsou často parametrizované a při obchodování s různými aktivy mohou fungovat různé kombinace hodnot.
Pro maximalizaci ziskovosti strategie a snížení rizik je zapotřebí rychlý a spolehlivý způsob optimalizace těchto parametrů \cite{efficient-backtesting}.

Nejjednodušší a časově nejnáročnější způsob, jak najít optimální nastavení, by bylo prohledat všechny možné kombinace.
Největší výhodou tohoto přístupu je, že zaručuje nalezení nejlepšího možného řešení, nicméně je v praxi nepoužitelný pro velké množství možností, protože nedokončí v přiměřeně krátké době \cite{efficient-backtesting}.

\chapter{Optimalizace}

\iffalse
Jednotlivé hodnoty jsou uloženy v instancích třídy \texttt{trading::fraction}, která je inspirována třídou \texttt{boost::rational}.
Je však mnohem jednodušší a méně obecná.
Předpokládá, na rozdíl od \texttt{boost::rational}, že zlomky, mezi kterými se provádějí aritmetické operace, mají stejného jmenovatele.
Porušení tohoto předpokladu vede k nedefinovanému chování.
Celá sekvence je uložena v instanci třídy \texttt{std::array}, což také znamená, že velikost sekvence musí být známa v době kompilace.
Pokud by to v budoucnu bylo nalezeno za nevhodné, může být nahrazena třídou \texttt{std::vector} nebo \texttt{boost::container::small\_vector} ke snížení počtu malých dynamických alokací.
\fi

\section{Návrh a generování stavového prostoru}
Pro optimalizaci byly vybrány 4 konfigurační proměnné: typ indikátoru, perioda indikátoru, úrovně a velikosti nákupů.
Hodnoty proměnných lze generovat buď systematicky, nebo náhodně.
Důležitým aspektem, který bylo nutné dodržet, bylo, že obě verze generátorů musí generovat hodnoty ze stejného vyhledávacího prostoru, jinak by byly výsledky jednotlivých optimalizačních metod nesrovnatelné.

Generátory sdílejí společné rozhraní, jakmile jsou vytvořeny, lze je použít voláním volacího operátoru třídy.
Náhodné generátory mají operátor přetížen.
Specializace přebírá jeden argument představující původní hodnotu používanou při generování následující hodnoty.
Systematické generátory vracejí korutiny a náhodné generátory vracejí přímo datové typy představující konfigurační proměnné.
Systematické generátory vracejí stejné hodnoty pouze jednou a náhodné generátory je mohou vrátit vícekrát.
V ideálním případě by měly náhodné generátory vracet stejné hodnoty se stejnou pravděpodobností.

\subsection{Typ indikátoru}
Typ indikátoru je reprezentován enumerační třídou \texttt{indicator\_tag} se dvěma možnými hodnotami: \texttt{indicator\_tag::ema} a \texttt{indicator\_tag::sma}.
Jediný způsob, jak parametrizovat tuto proměnnou, je použít buď jednu z hodnot, nebo obě.
Hodnoty jsou stejného typu, takže je lze uložit do pole.
Jelikož je jeho velikost malá a maximální velikost je známa v době kompilace, tak lze použít \texttt{etl::vector} k uložení hodnot pro generování.
Hodnoty lze náhodně rovnoměrně generovat pomocí instance třídy \tttext{std::uniform\_int\_distribution}.

\subsection{Perioda indikátoru}
SMA i EMA mají parametr perioda $p>1$.
Oba generátory generují hodnoty v intervalu $[f, t]$ s krokem $s$, kde $((f>t\implies((f-t) \mod s)=0 \land s<0) \land (t<f\implies((t-f) \mod s)=0 \land s>0)) \land s\neq0 \land t\neq f$.
Počet možných výstupních hodnot je roven $n=\frac{|t-f|}{s}+1$.

\subsubsection{Náhodné generování}
Při generování bez počáteční periody je nejprve pomocí \tttext{std::uniform\_int\_distribution} vygenerována hodnota $ v \in [\frac{p_{min}}{s}, = \frac{p_{max}}{s}$], kde $p_{min}=min(f, t) \land p_{max}=max(f, t)$, která je následně přeškálována $p_o=v\cdot s \land p_o \in [p_{min}, p_{max}]$.

Při generování hodnoty z počátku $o$ používá generátor navíc parametr $k\in[1,n/2]$.
Například, když $f=1 \land t=5 \land s=1 \land o=3 \land k=1$, pak jsou všechny možné výstupní hodnoty periody $ p_o \in [o-(k\cdot s), o+(k\cdot s)]=[2, 4]$.
Jediné úskalí nastává na okrajích intervalu, kdy mohou být hodnoty generovány z 2 podintervalů. Například, když $f=1 \land t=5 \land s=1 \land o=1 \land k=1$, pak jsou všechny možné výstupní periody $p_o \in [1, 2] \cup [5, 5]$.
Jelikož jsou periody generovány pomocí datového typu integer, který může nabývat i záporných hodnot, lze nejprve vygenerovat hodnotu $ w \in [o-(k\cdot s), o+(k\cdot s)]$, výslednou periodu získat $(w<p_{min}\implies(p_o=w+(n*|s|))) \land (w>p_{max}\implies(p_o=w-(n*|s|))) \land (w \in [p_{min}, p_{max}]\implies p_o=w)$.

\subsection{Úrovně nákupu}
Pro úrovně nákupu platí, že $\{l_0, l_1,\dots,l_n\} \in \mathbb{Q} \cap [0, 1] \land n>0 \land [\forall i \in \{0,\dots,n-1\} : l_i > l_{i+1}]$.
Příkladem takové sekvence racionálních čísel může být $\{\frac{1}{2}, \frac{1}{3}, \frac{1}{4}\}$.
Generátory jsou parametrizovány pomocí parametru $u\geq n$, který označuje počet jedinečných zlomků, které se mohou objevit v sekvenci.
Zlomky, které tvoří výslednou sekvenci, jsou brány z množiny $\{\frac{u}{d}, \frac{u-1}{d},\dots, \frac{1}{d}\} \in [0, 1] \land d=u+1$.
Pro $n=3 \land u=4$ jsou všechny možné platné sekvence: $\{\frac{4}{5}, \frac{3}{5}, \frac{2}{5}\}, \{\frac{4}{5}, \frac{3}{5}, \frac{1}{5}\}, \{\frac{4}{5}, \frac{2}{5}, \frac{1}{5}\}, \{\frac{3}{5}, \frac{2}{5}, \frac{1}{5}\}$.

\subsubsection{Systematické generování}
Sekvence jsou generovány rekurzivně (viz \ref{lst:systematic:levels_generator}), protože se počet smyček \texttt{for} liší v závislosti na velikosti sekvence $n$.
Smyčky jsou na sobě závislé, následující (hlubší) smyčka musí znát předchozí hodnotu čitatele.
Nejvnitřnější smyčka vrací výslednou sekvenci.

\begin{lstlisting}[caption={~Metody pro systematické generování úrovní nákupu},label={lst:systematic:levels_generator},captionpos=t,float,abovecaptionskip=-\medskipamount,belowcaptionskip=\medskipamount,language=C]
    value_type operator()()
    {
        co_yield generate<0>(this->denom_);
    }

private:
    template<std::size_t depth = n_levels>
    requires (depth==n_levels)
    value_type generate(std::size_t) { co_yield this->levels_; }
    
    template<std::size_t depth = 0>
    requires (depth<n_levels)
    value_type generate(std::size_t prev_num)
    {
        for (std::size_t num{--prev_num}; num>n_levels-depth-1; num--) {
            this->levels_[depth] = fraction_t{num, this->denom_};
            co_yield generate<depth+1>(num);
        }
    }
\end{lstlisting}

\subsubsection{Náhodné generování}
Instance třídy \texttt{trading::random::levels\_generator} má mezi svými atributy pole se všemi možnými hodnotami výstupní sekvence.
Při generování náhodné sekvence bez počátku nejprve tyto hodnoty zamíchá, seřadí prvních $n$ prvků v sestupném pořadí a nakonec je zkopíruje do výstupního pole.

Náhodné generování na základě počáteční sekvence je složitější.
Používá parametr $k \in [1, n] $, který je předán instanci při její konstrukci.
Udává počet prvků počáteční sekvence, které by měly být změněny.
Při generování je nejprve náhodně vybráno $n-k$ prvků, které budou zachovány z původní sekvence, a jsou uloženy do instance \texttt{etl::flat\_set}, což je kontejner pevné velikosti, který ukládá jedinečné prvky v určeném pořadí.
Dále je zamícháno pole se všemi možnými hodnotami.
Ve smyčce \texttt{while} se pak hodnoty pokouší vložit do sady s jedinečnými hodnotami, dokud není sada plná.
Nakonec jsou hodnoty překopírovány ze sady do výstupního pole.

\subsection{Velikosti nákupu}
Pro velikosti nákupu platí, že $ (\-s_0, s_1,\dots,s_n)\ \in \mathbb{Q} \cap [0, 1] \land n>1 \land \sum_{i=0}^{n} s_i = 1 $.
Příkladem takové sekvence může být $ (\frac{1}{4}, \frac{2}{4}, \frac{1}{4})\ $.
Generátory jsou parametrizovány pomocí parametru $u\geq 1$, který označuje počet jedinečných zlomků, které se mohou v sekvenci objevit.
Zlomky, které tvoří výslednou sekvenci, jsou brány z množiny $\{\frac{u}{d}, \frac{u-1}{d},\dots, \frac{1}{d}\} \in [0, 1] \land d=n+u-1$.
Pro $n=3 \land u=3$ jsou všechny možné platné sekvence: $\{\frac{1}{5}, \frac{1}{5}, \frac{3}{5}\}, \{\frac{1}{5}, \frac{2}{5}, \frac{2}{5}\}, \{\frac{1}{5}, \frac{3}{5}, \frac{1}{5}\}, \{\frac{2}{5}, \frac{1}{5}, \frac{2}{5}\}, \{\frac{2}{5}, \frac{2}{5}, \frac{1}{5}\}, \{\frac{3}{5}, \frac{1}{5}, \frac{1}{5}\}$.

\subsubsection{Systematické generování}
Velikosti nákupů \ref{lst:systematic:sizes_generator} jsou systematicky generovány podobně jako úrovně nákupů.
Používá se také rekurze, každá smyčka \texttt{for} předává součet zbývajících čitatelů následující smyčce.
Maximální lokální čitatel má buď hodnotu globálního maximálního čitatele, nebo součet zbývajících čitatelů snížený o jedna.

\begin{lstlisting}[caption={~Metody pro systematické generování velikostí nákupu},label={lst:systematic:sizes_generator},captionpos=t,float,abovecaptionskip=-\medskipamount,belowcaptionskip=\medskipamount,language=C]
    value_type operator()()
    {
        co_yield generate<0>(this->denom_);
    }

private:
    template<std::size_t depth = n_sizes>
    requires (depth+1==n_sizes)
    value_type generate(std::size_t remaining)
    {
        this->sizes_[depth] = fraction_t{remaining, this->denom_};
        co_yield this->sizes_;
    }

    template<std::size_t depth = 0>
    requires (depth+1<n_sizes)
    value_type generate(std::size_t remaining)
    {
        std::size_t max = (remaining>this->max_num_)
            ? this->max_num_ : remaining-1;
        for (std::size_t num{1}; num<=max; num++) {
            this->sizes_[depth] = fraction_t{num, this->denom_};
            co_yield generate<depth+1>(remaining-num);
        }
    }
\end{lstlisting}

\subsubsection{Náhodné generování}



 % include `text.tex' from `text/' subdirectory

\appendix\appendixinit % do not remove these two commands

\chapter{Přílohy}
\begin{lstlisting}[caption={~Společné části nastavení},label={lst:common:settings},captionpos=t,abovecaptionskip=-\medskipamount,belowcaptionskip=\medskipamount,language=C]
{
    "candles": {
        "count": 2160979,
        "currency pair": {
            "base": "ZRX",
            "quote": "USDT"
        },
        "from": 1551326400,
        "to": 1681230420
    },
    "optimization criterion": "prom",
    "resampling": {
        "averaging method": "ohlc4",
        "period[min]": 15
    },
    "search space": {
        "indicator": {
            "period": {
                "from": 3,
                "step": 3,
                "to": 105
            },
            "types": [
                "sma"
            ]
        },
        "levels": {
            "count": 3,
            "lower bound": "12/20",
            "unique count": 15
        },
        "open order sizes": {
            "unique count": 11
        }
    }
}
\end{lstlisting}

\begin{lstlisting}[caption={~Informace uložené o nejlepším stavu},label={lst:best-state},captionpos=t,abovecaptionskip=-\medskipamount,belowcaptionskip=\medskipamount]
{
    "configuration": {
        "indicator": {
            "period": 36,
            "type": "ema"
        },
        "levels": ["5/6","4/6","2/6"],
        "open sizes": ["5/7","1/7","1/7"]
    },
    "statistics": {
        "close balance": {
            "max": 78961.828125,
            "max drawdown": {
                "amount": -459.23828125,
                "percent": -1.4765747785568237
            },
            "max run up": {
                "amount": 68961.828125,
                "percent": 689.6182861328125
            },
            "min": 10000.0
        },
        "equity": {
            "max": 78961.828125,
            "max drawdown": {
                "amount": -11520.890625,
                "percent": -34.7911491394043
            },
            "max run up": {
                "amount": 69040.4375,
                "percent": 695.8746337890625
            },
            "min": 9921.390625
        },
        "gross loss": -459.23828125,
        "gross profit": 69421.0625,
        "net profit": 68961.828125,
        "open order counts": [24,2,0],
        "order ratio": 1.0833333333333333,
        "profit factor": 151.16566467285156,
        "total close orders": 24,
        "total open orders": 26
    }
},
\end{lstlisting} % include `appendix.tex' from `text/' subdirectory

\backmatter % do not remove this command

\printbibliography % print out the BibLaTeX-generated bibliography list

\include{text/medium} % include `medium.tex' from `text/' subdirectory

\end{document}
